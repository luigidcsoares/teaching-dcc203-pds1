% Created 2022-08-25 qui 07:19
% Intended LaTeX compiler: pdflatex
\documentclass[t, aspectratio=169]{beamer}
\usepackage[utf8]{inputenc}
\usepackage[T1]{fontenc}
\usepackage{graphicx}
\usepackage{longtable}
\usepackage{wrapfig}
\usepackage{rotating}
\usepackage[normalem]{ulem}
\usepackage{amsmath}
\usepackage{amssymb}
\usepackage{capt-of}
\usepackage{hyperref}
\usepackage[newfloat]{minted}
\usepackage{tikz}
\usetheme{default}
\author{Luigi D. C. Soares}
\date{DCC/UFMG (25/08/2022)}
\title{Apresentação do Curso}
\subtitle{Progamação e Desenvolvimento de Software I}
\title[Apresentação do Curso]{Apresentação do Curso}
\subtitle{Programação e Desenvolvimento de Software I}
\author[\tiny\{gleison.mendonca, luigi.domenico\}@dcc.ufmg.br]{%
Gleison S. D. Mendonça, Luigi D. C. Soares\texorpdfstring{\\}{}
\texttt{\{gleison.mendonca, luigi.domenico\}@dcc.ufmg.br}}
\institute[DCC/UFMG]{}
\date[25/08/2022]{}
%\usetheme{saori}
\usetheme{ufmg}
\hypersetup{
 pdfauthor={Luigi D. C. Soares},
 pdftitle={Apresentação do Curso},
 pdfkeywords={},
 pdfsubject={},
 pdfcreator={Emacs 28.1 (Org mode 9.6)}, 
 pdflang={English}}
\begin{document}

\maketitle


\begin{frame}[label={sec:orgfdcf03a}]{Objetivos}
\begin{itemize}
\item Introduzir o aluno aos conceitos de \alert{algoritmos} e \alert{estruturas de dados}
\end{itemize}
\end{frame}

\begin{frame}[label={sec:org4172ff5}]{Objetivos}
\begin{itemize}
\item Introduzir o aluno aos conceitos de \alert{algoritmos} e \alert{estruturas de dados}
\begin{itemize}
\item Noções da organização e funcionamento de um computador
\end{itemize}
\end{itemize}
\end{frame}

\begin{frame}[label={sec:orgededcc9}]{Objetivos}
\begin{itemize}
\item Introduzir o aluno aos conceitos de \alert{algoritmos} e \alert{estruturas de dados}
\begin{itemize}
\item Noções da organização e funcionamento de um computador
\item Noções de linguagens imperativas (vamos utilizar a linguagem C)
\end{itemize}
\end{itemize}
\end{frame}

\begin{frame}[label={sec:org2194848}]{Objetivos}
\begin{itemize}
\item Introduzir o aluno aos conceitos de \alert{algoritmos} e \alert{estruturas de dados}
\begin{itemize}
\item Noções da organização e funcionamento de um computador
\item Noções de linguagens imperativas (vamos utilizar a linguagem C)
\item Noções de estruturas de dados
\end{itemize}
\end{itemize}
\end{frame}

\begin{frame}[label={sec:org302879b}]{Por quê aprender programação?}
\begin{itemize}
\item Diversas áreas se beneficiam de programação
\end{itemize}
\end{frame}

\begin{frame}[label={sec:orgba4fdea}]{Por quê aprender programação?}
\begin{itemize}
\item Diversas áreas se beneficiam de programação
\begin{itemize}
\item Simulações, análise de dados, sistemas embutidos, sensores
\end{itemize}
\end{itemize}
\end{frame}

\begin{frame}[label={sec:orge217b59}]{Por quê aprender programação?}
\begin{itemize}
\item Diversas áreas se beneficiam de programação
\begin{itemize}
\item Simulações, análise de dados, sistemas embutidos, sensores
\item Matemática computacional
\end{itemize}
\end{itemize}
\end{frame}

\begin{frame}[label={sec:org4d55f85}]{Por quê aprender programação?}
\begin{itemize}
\item Diversas áreas se beneficiam de programação
\begin{itemize}
\item Simulações, análise de dados, sistemas embutidos, sensores
\item Matemática computacional
\item Física computacional
\end{itemize}
\end{itemize}
\end{frame}

\begin{frame}[label={sec:org593529c}]{Por quê aprender programação?}
\begin{itemize}
\item Diversas áreas se beneficiam de programação
\begin{itemize}
\item Simulações, análise de dados, sistemas embutidos, sensores
\item Matemática computacional
\item Física computacional
\item Química, Estatística, Economia, Linguística
\end{itemize}
\end{itemize}
\end{frame}

\begin{frame}[label={sec:org5e8ce5b}]{Organização do curso - Avaliação}
\begin{itemize}
\item Provas teóricas: 3 x 25 pontos
\item Laboratório: 10 pontos
\item Trabalhos práticos: 15 pontos
\end{itemize}
\end{frame}

\begin{frame}[label={sec:orgaaf569f}]{Organização do curso - Aulas}
\begin{itemize}
\item Aulas teóricas:
\begin{itemize}
\item Apresentação dos conceitos
\item Toda terça-feira
\end{itemize}
\end{itemize}
\end{frame}

\begin{frame}[label={sec:org6c6ab83}]{Organização do curso - Aulas}
\begin{itemize}
\item Aulas teóricas:
\begin{itemize}
\item Apresentação dos conceitos
\item Toda terça-feira
\end{itemize}

\item Aulas práticas:
\begin{itemize}
\item Toda quinta-feira
\item Suporte às aulas teóricas
\item Resolução de exercícios e dúvidas
\item Os exercícios serão liberados no início da aula e a entrega será até terça a noite
\item Soluções dos exercícios serão liberadas toda quarta
\end{itemize}
\end{itemize}
\end{frame}

\begin{frame}[label={sec:org0a163d7}]{Organização do curso - Laboratórios}
\begin{itemize}
\item Laboratórios Virtuais de Programação (VPL)
\begin{itemize}
\item Ambiente de programação integrado ao Moodle
\item Correção automática utilizando diversos casos de teste
\item Lembre-se sempre de apertar o botão \alert{avaliar}
\end{itemize}
\end{itemize}
\end{frame}

\begin{frame}[label={sec:org6d47566}]{Organização do curso - Trabalhos Práticos}
\begin{itemize}
\item Individual
\item Entrega no Moodle (VPL), correção automática com verificação de cópias
\item Julgamento do código:
\begin{itemize}
\item Facilidade de leitura
\item Comentários (se pertinente)
\end{itemize}
\end{itemize}
\end{frame}

\begin{frame}[label={sec:org2eecaaf}]{Organização do curso - Provas}
\begin{itemize}
\item Conteúdo base: livro, slides, exercícios
\item Faça todos os VPLs para praticar
\item Procure exercícios além dos VPLs (livro, internet)
\begin{itemize}
\item Codewars (\url{https://codewars.com})
\item Exercism (\url{https://exercism.org/tracks/c/exercises})
\item LeetCode (\url{https://leetcode.com/problemset/all/})
\item etc
\end{itemize}
\end{itemize}
\end{frame}

\begin{frame}[label={sec:org6121370}]{Bibliografia}
\begin{itemize}
\item Livro-texto: Linguagem C completa e descomplicada, André Backes

\item Outros:
\begin{itemize}
\item Introdução às Estruturas de Dados, Waldemar Celes
\item Projeto de Algoritmos com implementação em Pascal e C, 3a edição,
Nivio Ziviani
\item Algoritmos estruturados, 3a edição, Harry Farrer, Becker, Faria, Matos,
dos Santos, Maia
\end{itemize}
\end{itemize}
\end{frame}

\begin{frame}[label={sec:orgce48e31}]{Notas e Frequência}
\begin{itemize}
\item Terá lista de presença toda aula
\item Presença não é obrigatória, só venha se quiser

\item Se nota >= 60, aprovado!
\item Se nota < 60 e for infrequente (frequência < 75\%)
\begin{itemize}
\item Reprovado, conceito F
\item Não tem direito a exame especial
\end{itemize}

\item Exame especial: matéria do semestre inteiro!
\end{itemize}
\end{frame}

\begin{frame}[label={sec:org44fde7d}]{Dúvidas?}
\begin{itemize}
\item Fórum do Moodle (preferência)
\item E-mail:
\begin{itemize}
\item \href{mailto:gleison.mendonca@dcc.ufmg.br}{gleison.mendonca@dcc.ufmg.br} (Turmas \alert{TA1} e \alert{TF})
\item \href{mailto:luigi.domenico@dcc.ufmg.br}{luigi.domenico@dcc.ufmg.br} (Turmas \alert{TA2} e \alert{TM3})
\item Adicionar [PDS 1] ao assunto
\end{itemize}
\item Dúvidas complexas: podemos marcar um horário para atendimento individual
\end{itemize}
\end{frame}
\end{document}
