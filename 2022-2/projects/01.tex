% Created 2022-11-01 ter 10:40
% Intended LaTeX compiler: pdflatex
\documentclass[a4paper, 11pt]{article}
\usepackage[utf8]{inputenc}
\usepackage[T1]{fontenc}
\usepackage{graphicx}
\usepackage{longtable}
\usepackage{wrapfig}
\usepackage{rotating}
\usepackage[normalem]{ulem}
\usepackage{amsmath}
\usepackage{amssymb}
\usepackage{capt-of}
\usepackage{hyperref}
\usepackage[newfloat]{minted}
\usepackage[brazil, portuges]{babel}
\usepackage[utf8]{inputenc}
\usepackage{fancyhdr}
\usepackage[margin=1.2in]{geometry}
\usepackage[table, dvipsnames]{xcolor}
\usepackage{booktabs}
\usepackage{array}
\usepackage{enumitem}
\usepackage{datetime2}
\usepackage{tikz}
\usetikzlibrary{matrix, positioning}
\makeatletter
\DeclareRobustCommand*\course[1]{\gdef\@course{#1}}
\DeclareRobustCommand*\institution[1]{\gdef\@institution{#1}}
\DeclareRobustCommand*\semester[1]{\gdef\@semester{#1}}
\title{Trabalho Prático 1 - Ordenação}
\author{Profs. Gleison Mendonça e Luigi Soares}
\course{Programação e Desenvolvimento de Software I}
\institution{DCC / ICEx / UFMG}
\semester{2022.2}
\let\thetitle\@title{}
\let\theauthor\@author{}
\let\thecourse\@course{}
\let\theinstitution\@institution{}
\let\thesemester\@semester{}
\let\thedate\@date{}
\makeatother
\DTMnewdatestyle{brDateStyle}{%
\renewcommand{\DTMdisplaydate}[4]{##3/##2/##1}%
\renewcommand{\DTMDisplaydate}{\DTMdisplaydate}}
\DTMsetdatestyle{brDateStyle}
\pagestyle{fancy}
\fancyhf{}
\setlength{\headheight}{15pt}
\lhead{\theauthor \\ \thecourse}
\rhead{\theinstitution \\ \thesemester}
\rfoot{\thepage}
\hypersetup{
colorlinks,
linkcolor={red!50!black},
citecolor={blue!50!black},
urlcolor={blue!80!black}
}
\date{\today}
\title{}
\hypersetup{
 pdfauthor={},
 pdftitle={},
 pdfkeywords={},
 pdfsubject={},
 pdfcreator={Emacs 28.2 (Org mode 9.6)}, 
 pdflang={Portuges}}
\begin{document}

\begin{center}
\Large\bfseries\thetitle
\end{center}

Um algoritmo de ordenação é um algoritmo que coloca os elementos de uma lista
em uma determinada ordem (por exemplo, ascendente ou descendente). Formalmente,
o resultado de um algoritmo de ordenação deve satisfazer duas condições:

\begin{itemize}
\item A saída está em ordem monotônica (nenhum elemento é menor/maior que o anterior,
de acordo com a ordem requerida)
\item A saída é uma permutação da sequência original (i.e. mantém todos os elementos originais)
\end{itemize}

Vamos considerar uma ordem ascendente. O trabalho é dividido em 3 etapas:

\begin{enumerate}
\item Implementar uma função \textbf{swap} (troca), que recebe um arranjo \textbf{A} e duas posições
\textbf{i} e \textbf{j} e troca o valor de \textbf{A[i]} e \textbf{A[j]}. Esta função não possui nenhum retorno
(i.e. void). A troca deve ser feita no arranjo passado por parâmetro.

\item Implementar uma função \textbf{sorted} (ordenado), que recebe um arranjo \textbf{A}, o tamanho
\textbf{n} do arranjo e retorna verdadeiro se \textbf{A} está ordenado ou falso caso contrário.

\item Implementar o algoritmo de ordenação conhecido como \textbf{Bubble Sort}. Você \textbf{deve} implementar este
algoritmo em específico. A função \textbf{bubblesort} recebe um arranjo \textbf{A} e o tamanho \textbf{n} do arranjo.
A função não possui nenhum retorno (i.e. void). A ordenação deve ser feita no próprio arranjo
\textbf{A} passado por parâmetro para a função. \\

O Bubble Sort, ou ordenação por flutuação (literalmente, "por bolha"), é um algoritmo de ordenação
dos mais simples. A ideia é percorrer o arranjo diversas vezes, e a cada passagem fazer flutuar para
o topo o maior elemento da sequência. Para tanto, comparamos dois elementos adjacentes e, caso estejam
na ordem incorreta, os trocamos de lugar. Para exemplificar, considere o arranjo [5, 2, 8, 3, 7]:

\begin{center}
\begin{tikzpicture}[array/.style={
     draw, text=white, font=\bfseries, fill=NavyBlue, minimum width=2em,
     minimum height=2em, outer sep=0pt
 }]
     \matrix (A0) [matrix of math nodes, nodes={array, anchor=center}]
         {|[fill=RedOrange]|5 & |[fill=RedOrange]|2 & 8 & 3 & 7 \\};
     \node[right = 0.5cm of A0, font=\bfseries] (swap0) {5 > 2, troca!};

     \matrix[below = 0.25cm of A0] (A1) [matrix of math nodes, nodes={array, anchor=center}]
         {2 & |[fill=RedOrange]|5 & |[fill=RedOrange]|8 & 3 & 7 \\};
     \node[right = 0.5cm of A1, font=\bfseries] (swap1) {5 < 8, não troca!};

     \matrix[below = 0.25cm of A1] (A2) [matrix of math nodes, nodes={array, anchor=center}]
         {2 & 5 & |[fill=RedOrange]|8 & |[fill=RedOrange]|3 & 7 \\};
     \node[right = 0.5cm of A2, font=\bfseries] (swap2) {8 > 3, troca!};

     \matrix[below = 0.25cm of A2] (A3) [matrix of math nodes, nodes={array, anchor=center}]
         {2 & 5 & 3 & |[fill=RedOrange]|8 & |[fill=RedOrange]|7 \\};
     \node[right = 0.5cm of A3, font=\bfseries] (swap2) {8 > 7, troca!};

     \matrix[below = 0.25cm of A3] (A4) [matrix of math nodes, nodes={array, anchor=center}]
         {2 & 5 & 3 & 7 & |[fill=ForestGreen]|8 \\};
\end{tikzpicture}
\end{center}

Note que, ao final das trocas, o maior elemento (8) foi movido para a posição
 correta.  O restante do arranjo (2, 5, 3, 7), no entanto, ainda não está
 ordenado. Encontramos a posição do maior elemento. Agora, devemos repetir o
 processo para o subarranjo que ainda não está ordenado, para encontrar a
 posição do segundo maior elemento, do terceiro maior elemento e assim por
 diante. Quando podemos parar? Pense nisso!
\end{enumerate}
\end{document}
